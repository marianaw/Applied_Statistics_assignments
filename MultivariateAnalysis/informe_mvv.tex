\documentclass[a4paper,10pt]{article}
\usepackage[utf8]{inputenc}
\usepackage{graphicx}
\usepackage{subfig}
\usepackage{amsmath}

%opening
\title{Trabajo Práctico 1 Análisis Multivariado}
\author{Mariana Vargas V.}

\begin{document}

\maketitle

\section{Introducción}

Trabajamos realizando un análisis estadístico de una base de datos que consiste en datos recopilados acerca
de ciertas variables económicas de algunos países, extraídos de la página del Banco Mundial.

\section{Ejercicio 1}

Realizamos un análisis exploratorio de los datos a partir de medidas descriptivas y gráficos. Veamos por ejemplo el box-plot de las variables de la
figura 1.

\begin{figure}[h]
\centering
\includegraphics[width=8cm]{box_plot_all}
\caption{Boxplots para todas las variables.}
\end{figure}

Notar que las variables que corresponden a importación y exportación de servicios, y participación del valor agregado industrial tienen las medias más
altas, mientras que la tasa de crecimiento del PBI y el crecimiento poblacional tienen las más bajas. Podemos confirmarlo en la tabla 1.

\begin{table}[ht]
\centering
\begin{tabular}{rlllllllll}
  \hline
 &    expbser &     tcrec &    expaltec &    impbser &   partVAind &    Crecpob &      Inv & \\ 
  \hline
1 &Min.   :10.66   & -0.790   &  0.100   &  9.34   &  9.44   & -0.440   &  15.85 \\ 
  2 & 1st Qu.:17.13   &3.095   &  7.705   & 22.91   & 26.63   &  0.250   & 20.34  \\ 
  3 & Median :28.09   &3.865   & 17.130   & 29.39   & 28.27   &  0.875   & 21.50  \\ 
  4  & Mean   :31.41   &4.131   & 15.649   & 32.69   & 28.77   &  1.000   & 22.25  \\ 
  5 & 3rd Qu.:39.84   & 5.070   & 22.332   & 33.63   & 30.36   &  1.450   & 23.46  \\ 
  6 & Max.   :93.65   &8.000   &33.520   &97.74   &50.22   & 2.640   &36.33 \\ 
   \hline
\end{tabular}
\caption{Medidas descriptivas de las variables.}
\end{table}

Podemos también observar los histogramas para cada variable, lo cual se detalla en la figura 2. Podemos notar como muchas de las variables son bastante
sesgadas0.

\begin{figure}[ht]
\centering
\includegraphics[width=8cm]{hist_all}
\caption{Histogramas para todas las variables.}
\end{figure}

Analizando las matrices de varianzas y covarianzas, y la de correlaciones entre variables de las tablas 2 y 3 podemos extraer algunas conclusiones
interesantes. En primer lugar, las exportaciones e importaciones tienen una alta varianza, incluyendo la exportación de alta tecnología (en menor medida) y 
una correlación positiva entre sí, mientras que variables como el crecimiento poblacional y la exportación de servicios o tecnología mantienen una 
correlación negativa. Esto sugiere que países con industrias de alto valor agregado que exportan productos de alta tecnología no tienen un crecimiento
poblacional marcado, y viceversa.

\begin{table}[ht]
\centering
\begin{tabular}{rrrrrrrr}
  \hline
 & expbser & tcrec & expaltec & impbser & partVAind & Crecpob & Inv \\ 
  \hline
expbser & 422.28 & 12.92 & 32.20 & 395.67 & 18.01 & -8.15 & 19.33 \\ 
  tcrec & 12.92 & 3.68 & 3.60 & 14.65 & 1.95 & 0.05 & 4.78 \\ 
  expaltec & 32.20 & 3.60 & 99.66 & 13.59 & -1.78 & -4.93 & -2.49 \\ 
  impbser & 395.67 & 14.65 & 13.59 & 400.17 & 0.42 & -4.70 & 20.40 \\ 
  partVAind & 18.01 & 1.95 & -1.78 & 0.42 & 44.94 & -1.82 & 20.61 \\ 
  Crecpob & -8.15 & 0.05 & -4.93 & -4.70 & -1.82 & 0.72 & -0.17 \\ 
  Inv & 19.33 & 4.78 & -2.49 & 20.40 & 20.61 & -0.17 & 19.88 \\ 
   \hline
\end{tabular}
\caption{Matriz de varianzas y covarianzas.}
\end{table}

\begin{table}[ht]
\centering
\begin{tabular}{rrrrrrrr}
  \hline
 & expbser & tcrec & expaltec & impbser & partVAind & Crecpob & Inv \\ 
  \hline
expbser & 1.00 & 0.33 & 0.16 & 0.96 & 0.13 & -0.47 & 0.21 \\ 
  tcrec & 0.33 & 1.00 & 0.19 & 0.38 & 0.15 & 0.03 & 0.56 \\ 
  expaltec & 0.16 & 0.19 & 1.00 & 0.07 & -0.03 & -0.58 & -0.06 \\ 
  impbser & 0.96 & 0.38 & 0.07 & 1.00 & 0.00 & -0.28 & 0.23 \\ 
  partVAind & 0.13 & 0.15 & -0.03 & 0.00 & 1.00 & -0.32 & 0.69 \\ 
  Crecpob & -0.47 & 0.03 & -0.58 & -0.28 & -0.32 & 1.00 & -0.04 \\ 
  Inv & 0.21 & 0.56 & -0.06 & 0.23 & 0.69 & -0.04 & 1.00 \\ 
   \hline
\end{tabular}
\caption{Matriz de correlaciones.}
\end{table}

Para un análisis más preciso hay que tener en cuenta que Estonia y China producen valores atípicos detectados a través del cómputo de las distancias
de Mahalanobis, que para estos países arroja valores de $14.39$ y $14.42$ respectivamente.

\section{Ejercicio 2}

Nos concentramos ahora en estudiar la distribución de estas variables. Nuestra pregunta es si son variables normales, lo que abordamos a través de 
pruebas de hipótesis y pruebas gráficas, tanto univariadas como multivariadas.
Entre las primeras podemos trabajar con qqplots para cada una de las variables, lo que esta sumarizado en la figura 4. A simple vista podemos notar que
las importaciones y exportaciones de bienes y servicios, las inversiones, y la participación del valor agregado no tienen una distribución normal.
Para confirmar esto realizamos un test de Shapiro-Wilks: para las exportaciones e importaciones de bienes y servicios obtenemos p-valores de $0.0002$ y
$0.0001$ respectivamente, para la participación del VA obtenemos $0.001$ y, por último, $0.0068$ para las inversiones, confirmando así nuestra hipótesis.
Las demás variables tienen una distribución normal.

\begin{figure}[ht]
\centering
\includegraphics[width=7cm]{mahalanobis}
\caption{qq-plot de las distancias de Mahalanobis ordenadas.}
\end{figure}


\begin{figure}[ht]
\begin{tabular}{cc}
 \includegraphics[width=55mm]{expbser_qq} &   \includegraphics[width=55mm]{tcrec_qq} \\[6pt]
 \includegraphics[width=55mm]{expaltec_qq} &   \includegraphics[width=55mm]{impbser_qq} \\[6pt]
 \includegraphics[width=55mm]{partVAind_qq} &   \includegraphics[width=55mm]{crecpob_qq} \\[6pt]
 \includegraphics[width=55mm]{inv_qq}
\end{tabular}
\caption{QQ-plots para todas las variables. De izquierda a derecha: exportaciones de bs y serv., tasa de crecimiento, exportación de tecnología, 
importaciones, participación del VA, crecimiento poblacional, e inversiones.}
\end{figure}

Entre las pruebas multivariadas podemos usar un qqplot de las distancias de Mahalanobis ordenadas (figura 3), que bajo una distribución normal se 
distribuyen como una $\chi^2$. El gráfico de los cuantiles empíricos versus los de la $\chi^2$ con siete grados de libertad sugiere que la distribución es
normal multivariada.
Para completar nuestro análisis calculamos la asimetría y kurtosis, obteniendo $0.34$ y $1.586$ respectivamente, confirmando así que la distribución es
una normal multivariada.


\section{Ejercicio 3}
Se nos pide realizar una prueba de hipótesis para verificar si el vector dado $mu_h0 = (30, 4, 15, 30, 27, 0.8, 20)$ es efectivamente el vector de medias
de la población.
Calculamos entonces el estadístico $T^2$ de Hotelling obteniendo $T^2 = 18.57$ mientras que el valor crítico para una confianza del $95\%$ es $23.42$. Luego
aceptamos la hipótesis nula de que $mu_h0$ es el vector verdadero de medias.

\section{Ejercicio 4}

Realizamos el análisis en dos grupos separados: países desarrollados y países subdesarrollados.

Vemos las medidas descriptivas para los países del primer grupo en la tabla 4 y las del segundo en la tabla 5.
\begin{table}[ht]
\centering
\begin{tabular}{rlllllll}
  \hline
 &    expbser &     tcrec &    expaltec &    impbser &   partVAind &    Crecpob &      Inv \\ 
  \hline
1 & Min.   :10.76   &1.750   & 7.64   & 9.34   &24.45   &0.0800   &17.31   \\ 
  2 & 1st Qu.:27.90   & 3.080   & 15.22   & 27.31   & 26.71   & 0.1500   & 20.59   \\ 
  3 & Median :30.14   &3.720   &19.31   &32.41   &28.96   &0.2500   &20.87   \\ 
  4 & Mean   :35.40   & 3.593   & 20.39   & 33.53   & 28.73   & 0.4785   & 21.27   \\ 
  5 & 3rd Qu.:45.83   & 4.180   & 27.33   & 40.20   & 30.23   & 0.7400   & 21.55   \\ 
  6 & Max.   :85.50   &5.530   &33.52   &82.29   &34.49   &1.2700   &26.15   \\ 
   \hline
\end{tabular}
\caption{Medidas descriptivas de países desarrollados.}
\end{table}

\begin{table}[ht]
\centering
\begin{tabular}{rlllllll}
  \hline
 &    expbser &     tcrec &    expaltec &    impbser &   partVAind &    Crecpob &      Inv \\ 
  \hline
1 & Min.   :10.66   & -0.790   &  0.10   & 11.52   &  9.44   & -0.440   & 15.85   \\ 
  2 & 1st Qu.:15.08   & 3.610   & 3.61   &18.02   &26.60   & 1.230   &20.26   \\ 
  3 & Median :23.60   &  4.400   &  7.90   & 28.76   & 28.01   &  1.460   & 22.49   \\ 
  4 & Mean   :27.43   &  4.669   & 10.91   & 31.86   & 28.81   &  1.522   & 23.24   \\ 
  5 & 3rd Qu.:29.75   &  6.570   & 18.58   & 31.53   & 32.08   &  2.240   & 24.51   \\ 
  6 & Max.   :93.65   &  8.000   & 29.84   & 97.74   & 50.22   &  2.640   & 36.33   \\ 
   \hline
\end{tabular}
\caption{Medidas descriptivas de países subdesarrollados.}
\end{table}

Notar como las medias de las tasas de crecimiento son sistemáticamente más bajas en los países subdesarrollados
(lo podemos ver prestando atención a las medianas, que son una medida más robusta con respecto a los valores atípicos), así también como los niveles de exportación
tanto de bienes y servicios como de tecnología. En cambio, el crecimiento poblacional tiende a ser mayor. 

\begin{table}[ht]
\centering
\begin{tabular}{rrrrrrrr}
  \hline
 & expbser & tcrec & expaltec & impbser & partVAind & Crecpob & Inv \\ 
  \hline
expbser & 369.49 & 6.37 & -74.90 & 331.07 & 4.60 & -2.35 & -10.28 \\ 
  tcrec & 6.37 & 0.91 & 1.02 & 4.47 & 1.13 & -0.06 & -0.29 \\ 
  expaltec & -74.90 & 1.02 & 72.87 & -73.75 & -0.36 & 0.21 & -6.16 \\ 
  impbser & 331.07 & 4.47 & -73.75 & 305.99 & -0.44 & -1.74 & -8.59 \\ 
  partVAind & 4.60 & 1.13 & -0.36 & -0.44 & 7.72 & -0.71 & 1.85 \\ 
  Crecpob & -2.35 & -0.06 & 0.21 & -1.74 & -0.71 & 0.17 & 0.11 \\ 
  Inv & -10.28 & -0.29 & -6.16 & -8.59 & 1.85 & 0.11 & 5.82 \\ 
   \hline
\end{tabular}
\caption{Matriz de covarianzas de países desarrollados.}
\end{table}

\begin{table}[ht]
\centering
\begin{tabular}{rrrrrrrr}
  \hline
 & expbser & tcrec & expaltec & impbser & partVAind & Crecpob & Inv \\ 
  \hline
expbser & 475.92 & 25.20 & 101.07 & 486.04 & 33.24 & -10.13 & 59.04 \\ 
  tcrec & 25.20 & 6.13 & 12.00 & 27.01 & 2.88 & -0.44 & 9.10 \\ 
  expaltec & 101.07 & 12.00 & 86.02 & 93.50 & -2.96 & -5.12 & 11.09 \\ 
  impbser & 486.04 & 27.01 & 93.50 & 526.20 & 1.38 & -7.10 & 52.86 \\ 
  partVAind & 33.24 & 2.88 & -2.96 & 1.38 & 85.91 & -3.12 & 41.02 \\ 
  Crecpob & -10.13 & -0.44 & -5.12 & -7.10 & -3.12 & 0.74 & -1.57 \\ 
  Inv & 59.04 & 9.10 & 11.09 & 52.86 & 41.02 & -1.57 & 33.49 \\ 
   \hline
\end{tabular}
\caption{Matriz de covarianzas de países subdesarrollados.}
\end{table}

\begin{table}[ht]
\centering
\begin{tabular}{rrrrrrrr}
  \hline
 & expbser & tcrec & expaltec & impbser & partVAind & Crecpob & Inv \\ 
  \hline
expbser & 1.00 & 0.35 & -0.46 & 0.98 & 0.09 & -0.30 & -0.22 \\ 
  tcrec & 0.35 & 1.00 & 0.12 & 0.27 & 0.43 & -0.15 & -0.13 \\ 
  expaltec & -0.46 & 0.12 & 1.00 & -0.49 & -0.02 & 0.06 & -0.30 \\ 
  impbser & 0.98 & 0.27 & -0.49 & 1.00 & -0.01 & -0.24 & -0.20 \\ 
  partVAind & 0.09 & 0.43 & -0.02 & -0.01 & 1.00 & -0.63 & 0.28 \\ 
  Crecpob & -0.30 & -0.15 & 0.06 & -0.24 & -0.63 & 1.00 & 0.12 \\ 
  Inv & -0.22 & -0.13 & -0.30 & -0.20 & 0.28 & 0.12 & 1.00 \\ 
   \hline
\end{tabular}
\caption{Matriz de correlaciones de países desarrollados.}
\end{table}

\begin{table}[ht]
\centering
\begin{tabular}{rrrrrrrr}
  \hline
 & expbser & tcrec & expaltec & impbser & partVAind & Crecpob & Inv \\ 
  \hline
expbser & 1.00 & 0.47 & 0.50 & 0.97 & 0.16 & -0.54 & 0.47 \\ 
  tcrec & 0.47 & 1.00 & 0.52 & 0.48 & 0.13 & -0.21 & 0.64 \\ 
  expaltec & 0.50 & 0.52 & 1.00 & 0.44 & -0.03 & -0.64 & 0.21 \\ 
  impbser & 0.97 & 0.48 & 0.44 & 1.00 & 0.01 & -0.36 & 0.40 \\ 
  partVAind & 0.16 & 0.13 & -0.03 & 0.01 & 1.00 & -0.39 & 0.76 \\ 
  Crecpob & -0.54 & -0.21 & -0.64 & -0.36 & -0.39 & 1.00 & -0.32 \\ 
  Inv & 0.47 & 0.64 & 0.21 & 0.40 & 0.76 & -0.32 & 1.00 \\ 
   \hline
\end{tabular}
\caption{Matriz de correlaciones de países subdesarrollados.}
\end{table}

A partir de las matrices de covarianzas y correlaciones de las tablas 6, 7, 8 y 9 podemos ver cómo en los países desarrollados las variables presentan
mayor varianza. También observamos que en los países desarrollados la exportación de bienes y servicios tiene una relación negativa respecto de la 
exportación de alta tecnología mientras que en los países subdesarrollados esta relación es positiva. En ambos casos la exportación de tecnología está 
asociado positivamente a una tasa de crecimiento, lo que sugiere que la industria de alto valor agregado es un motor de crecimiento.
Para enriquecer nuestro análisis observamos los gráficos de las figuras 5, 6, 7 y 8. Notamos que las tasas de crecimiento no son muy distintas, excepto por
el hecho de que en los países desarrollados es seis veces menos variable. Lo que es interesante respecto de esta variable es cómo se relaciona con las 
demás acorde al tipo de país.

\begin{figure}[ht]
\centering
\includegraphics[width=8cm]{des_qq}
\caption{qq-plot de los países desarrollados.}
\end{figure}

\begin{figure}[ht]
\centering
\includegraphics[width=8cm]{sub_qq}
\caption{qq-plot de los países subdesarrollados.}
\end{figure}

\begin{figure}[ht]
\centering
\includegraphics[width=8cm]{des_hist}
\caption{Histogramas de los países desarrollados.}
\end{figure}

\begin{figure}[ht]
\centering
\includegraphics[width=8cm]{sub_hist}
\caption{Histogramas de los países subdesarrollados.}
\end{figure}

\section{Ejercicio 5}

Hicimos un test de verosimilitud para corroborar que la diferencia entre matrices de covarianza por grupo es significativamente distinta.
Planteamos H0 = ``las matrices de covarianza son iguales'' contra H1 = ``las matrices de covarianza son distintas''. El p-valor es de $1.3 * e^{-6}$, con 
lo que tenemos evidencia suficiente para rechazar la hipótesis nula y afirmar que las matrices son significativamente distintas para países desarrollados
y subdesarrollados.

\section{Ejercicio 6}

La descomposición espectral de la matriz arrojó el siguiente vector de valores propios:

\begin{equation*}
 D = (810.3787144, 100.5699330,  58.3808456,  17.0778631,   3.2166448,   1.5382517,   0.1624454)
\end{equation*}

\section{Ejercicio 7}
Para estandarizar la matriz las matrices $D$ y $U$ involucradas en la descomposición espectral de $S$, calculadas con el comando \verb svd .
La ecuación es
\begin{equation*}
 X_1 = X U D^{-1/2} U'
\end{equation*}

La matriz resultante se muestra en la tabla 10.

\begin{table}[ht]
\centering
\begin{tabular}{rrrrrrrr}
  \hline
 & 1 & 2 & 3 & 4 & 5 & 6 & 7 \\ 
  \hline
1 & -0.89 & -2.83 & 1.44 & 1.64 & 4.41 & 4.25 & 2.09 \\ 
  2 & 0.15 & -2.39 & 1.99 & 1.33 & 3.16 & 5.64 & 4.15 \\ 
  3 & 2.52 & -1.62 & 1.10 & 3.30 & 3.58 & 5.25 & 3.08 \\ 
  4 & -0.77 & -0.32 & 2.36 & 1.38 & 3.90 & 5.05 & 3.41 \\ 
  5 & 1.36 & -0.50 & 2.14 & 1.53 & 3.82 & 6.37 & 3.08 \\ 
  6 & 0.13 & -0.03 & 0.71 & 1.81 & 5.06 & 5.03 & 2.70 \\ 
  7 & -0.38 & 0.62 & 2.38 & 1.75 & 6.77 & 5.15 & 5.44 \\ 
  8 & 1.31 & -0.76 & 3.25 & 5.20 & 3.91 & 3.97 & 4.10 \\ 
  9 & -1.65 & 0.62 & 2.19 & 3.12 & 1.55 & 4.98 & 2.63 \\ 
  10 & 1.17 & 0.42 & 3.00 & 1.30 & 4.99 & 5.71 & 2.30 \\ 
  11 & 0.19 & -1.07 & 2.81 & 1.57 & 3.28 & 4.73 & 3.64 \\ 
  12 & -0.05 & -1.56 & 2.17 & 2.25 & 4.30 & 3.73 & 3.26 \\ 
  13 & -0.86 & -0.77 & 1.35 & 2.54 & 3.04 & 6.64 & 2.63 \\ 
  14 & -1.04 & -1.76 & 0.66 & 2.87 & 5.83 & 6.36 & 2.26 \\ 
  15 & -0.99 & -1.55 & 0.86 & 4.32 & 4.42 & 6.34 & 5.08 \\ 
  16 & 0.04 & -0.57 & 0.90 & 0.81 & 3.16 & 5.35 & 4.20 \\ 
  17 & 0.03 & -0.73 & 1.21 & 1.78 & 4.06 & 2.62 & 2.85 \\ 
  18 & -0.65 & -2.17 & 3.39 & 1.09 & 3.97 & 4.29 & 5.11 \\ 
  19 & -0.07 & 0.38 & 2.69 & 2.10 & 3.82 & 6.12 & 3.68 \\ 
  20 & -0.37 & -0.93 & 0.79 & 1.47 & 3.84 & 4.70 & 3.10 \\ 
  21 & -0.01 & -0.98 & 1.14 & 2.03 & 3.78 & 3.70 & 4.46 \\ 
  22 & 0.99 & -0.21 & 2.47 & 1.87 & 4.33 & 4.87 & 2.08 \\ 
  23 & -1.13 & -1.13 & 3.76 & 3.01 & 4.95 & 4.35 & 1.52 \\ 
  24 & -1.22 & -1.31 & 4.01 & 1.95 & 3.56 & 6.10 & 3.54 \\ 
  25 & -0.58 & 1.29 & 0.86 & 2.34 & 3.30 & 4.06 & 3.90 \\ 
  26 & 1.42 & -1.60 & 2.21 & 1.50 & 3.24 & 5.83 & 3.67 \\ 
   \hline
\end{tabular}
\caption{Matriz de datos estandarizada.}
\end{table}

\end{document}