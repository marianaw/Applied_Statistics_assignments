\documentclass[a4paper,10pt]{article}
\usepackage[utf8]{inputenc}
\usepackage{amsmath}
\usepackage{graphicx}
\usepackage{subfig}
\usepackage{listings}
\usepackage{verbatim}

%\documentclass[a4paper,10pt]{scrartcl}

\title{Trabajo práctico Nº2}
\author{Mariana Vargas V.}
\date{\today}

\pdfinfo{%
  /Title    ()
  /Author   ()
  /Creator  ()
  /Producer ()
  /Subject  ()
  /Keywords ()
}

\begin{document}
\maketitle

\section{Análisis de Componentes Principales}

El Análisis de Componentes Principales es una técnica multivariada que consiste en una transformación ortogonal de una matrix de datos en otra cuyas
variables resulten linealmente independientes. Estas, llamadas \textit{componentes principales}, son tales que representan las direcciones de mayor 
variabilidad de los datos, permitiendo así la posibilidad de reducir la dimensión.

\subsection{Nuestros datos}

Llevamos adelante el análisis de componentes principales en la base de datos \verb indice.RData  que tiene información sobre las finanzas de un grupo
de empresas.
Con el comando \verb prcomp  de R obtuvimos las componentes principales. La salida de \verb summary(pc)  sugiere que las dos primeras son las más importantes
dado que representan el $99\%$ de la varianza.

\verbatiminput{summary_pc.txt}

Podemos confirmar esto en la figura 1 en donde se muestran un gráfico de barra de las componentes contra la varianza.

\begin{figure}[h]
\centering
\includegraphics[width=8cm]{biplot_pc}
\caption{Componentes vs. la varianza.}
\end{figure}

En la figura 2 podemos ver a las variables representadas en función de las dos primeras componentes.

\begin{figure}[h]
\centering
\includegraphics[width=10cm]{biplot_pc2}
\caption{Biplot en las dos primeras componentes.}
\end{figure}

Observamos que algunas variables se aproximan a la primera componente, tales como la rentabilidad económica, el margen de explotación y el costo marginal
de financiamiento, y otras que se aproximan a la segunda componente, como la inmovilización del activo. También hay variables que mantienen con las 
componentes una relación negativa, como la inmovilización del patrimonio, el pasivo, y la solvencia.
La matriz de correlaciones (tabla 1) confirma esto.

\begin{table}[ht]
\centering
\begin{tabular}{rrrrrrrrrrr}
  \hline
 & 1 & 2 & 3 & 4 & 5 & 6 & 7 & 8 & 9 & 10 \\ 
  \hline
LIQACID & -0.78 & -0.62 & 0.03 & 0.01 & -0.02 & -0.01 & -0.00 & -0.00 & -0.00 & 0.00 \\ 
  SOLVENC & -0.33 & -0.94 & 0.10 & 0.03 & -0.00 & 0.00 & 0.00 & 0.00 & -0.00 & 0.00 \\ 
  PROPACT & 0.94 & -0.33 & 0.03 & 0.03 & -0.00 & 0.01 & 0.00 & -0.00 & 0.00 & -0.00 \\ 
  PNOCOR & -0.97 & 0.21 & -0.03 & -0.01 & -0.06 & 0.00 & 0.00 & 0.00 & 0.00 & -0.00 \\ 
  AUTOFIN & 1.00 & 0.08 & 0.04 & -0.02 & -0.01 & -0.01 & -0.00 & 0.00 & 0.00 & -0.00 \\ 
  INMACT & -0.34 & 0.93 & 0.08 & 0.09 & -0.00 & -0.00 & -0.00 & 0.00 & -0.00 & 0.00 \\ 
  INMPN & -0.90 & 0.41 & 0.11 & -0.08 & 0.01 & 0.00 & -0.00 & -0.00 & 0.00 & -0.00 \\ 
  RENTECO & 0.99 & 0.12 & 0.04 & -0.01 & -0.01 & -0.00 & 0.00 & -0.00 & -0.00 & -0.00 \\ 
  MAREXP & 1.00 & 0.06 & 0.01 & -0.02 & -0.03 & 0.01 & -0.01 & 0.00 & 0.00 & 0.00 \\ 
  REXP\_INT & 0.99 & 0.15 & 0.05 & -0.01 & -0.01 & 0.00 & 0.01 & -0.00 & -0.00 & 0.00 \\ 
   \hline
\end{tabular}
\caption{Matriz de correlaciones entre componentes y variables usando la matriz S.}
\end{table}

Repetimos el análisis usando la matriz de correlación de los datos.
La salida muestra que esta vez las cuatro primeras componentes suman el $95 \%$ de la correlación entre las variables, lo cual podemos ver en la 
figura 3.

\verbatiminput{summary_pc_corr.txt}

\begin{figure}[h]
\centering
\includegraphics[width=8cm]{screeplot_corr}
\caption{Gráfico de las componentes vs. la varianza, usando la matriz de correlación.}
\end{figure}

La figura 4 muestra las variables originales en función de las componentes elegidas, que a diferencia del análisis anterior, maximizan la correlación entre
variables y no su varianza.

\begin{figure}[h]
\centering
\includegraphics[width=14cm]{biplot_pca_corr}
\caption{Gráfico de las variables en función de las componentes principales.}
\end{figure}

\begin{table}[ht]
\centering
\begin{tabular}{rrrrrrrrrrr}
  \hline
 & 1 & 2 & 3 & 4 & 5 & 6 & 7 & 8 & 9 & 10 \\ 
  \hline
LIQACID & -0.75 & -0.65 & 0.04 & 0.01 & 0.00 & 0.12 & 0.05 & 0.04 & 0.02 & 0.00 \\ 
  SOLVENC & -0.81 & -0.57 & -0.06 & 0.04 & -0.04 & 0.06 & 0.04 & 0.03 & -0.02 & 0.00 \\ 
  PROPACT & -0.91 & -0.28 & -0.05 & 0.29 & -0.01 & -0.07 & -0.04 & -0.01 & 0.00 & -0.00 \\ 
  PNOCOR & 0.88 & 0.07 & 0.34 & 0.23 & 0.04 & 0.20 & -0.05 & 0.03 & -0.00 & -0.00 \\ 
  AUTOFIN & -0.74 & 0.45 & -0.18 & -0.19 & 0.42 & 0.05 & -0.03 & 0.04 & -0.00 & -0.00 \\ 
  INMACT & 0.86 & 0.03 & -0.11 & 0.44 & 0.21 & -0.09 & 0.07 & 0.03 & 0.00 & 0.00 \\ 
  INMPN & 0.94 & 0.08 & -0.06 & -0.30 & -0.04 & 0.04 & 0.08 & 0.01 & -0.00 & -0.00 \\ 
  RENTECO & -0.72 & 0.65 & 0.05 & 0.17 & 0.02 & 0.11 & 0.07 & -0.08 & -0.00 & -0.00 \\ 
  MAREXP & -0.67 & 0.40 & 0.61 & -0.06 & -0.00 & -0.11 & 0.03 & 0.04 & -0.00 & -0.00 \\ 
  REXP\_INT & -0.37 & 0.83 & -0.29 & 0.12 & -0.28 & 0.04 & -0.00 & 0.07 & 0.00 & 0.00 \\ 
   \hline
\end{tabular}
\caption{Matriz de correlación de las variables en función de las componentes.}
\end{table}

De la matriz de correlación entre componentes y variables muestra una relación positiva entre la inmovilización del activo y del patrimonio, y del pasivo
respecto de la primera componente, mientras que la propiedad del activo mantiene una relación negativa.

\section{Análisis de correspondencia}

Mediante el análisis de correspondencia estudiamos tablas de contingencia de variables clasificatorias. Esta técnica es análoga al análisis de componentes
principales.

\subsection{Resultados en los datos}

Trabajamos con una base de datos que tiene información sobre una encuensta sobre hogares. Analizamos las variables correspondientes al estado civil y al
nivel de educación.
Con el comando \verb ca  obtenemos el análisis de correspondencia. La raíz cuadrada de la suma de los autovalores nos da un índice de correlación
entre filas y columnas. En este
caso es de $0.4$, lo que indica una baja correlación. Esto es, el nivel de educación no necesariamente define el estado civil, y viceversa.
En la figura 5 podemos visualizar los resultados. Recordemos que puntos correspondientes a las filas 
que estén más cerca indican perfiles columna más parecidos. Por ejemplo, observar que las personas unidas y casadas tienden a tener el mismo nivel de 
educación, contrario a lo que ocurre con personas viudas y solteras, que se ven alejadas en el gráfico.

\begin{figure}[h]
\centering
\includegraphics[width=10cm]{corr_simple}
\caption{Gráfico de correspondencia simple entre estado civil y educación.}
\end{figure}

La función \verb mjca  nos permite hacer un análisis de correspondencia múltiple. En este caso la inercia indica una correlación del $84\%$ entre las
variables fila y las variables columna.

\begin{figure}[h]
\centering
\includegraphics[width=12cm]{corr_mult}
\caption{Gráfico de correspondencia simple entre estado civil y educación.}
\end{figure}

\end{document}
