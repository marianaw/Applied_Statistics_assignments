\documentclass[a4paper,10pt]{article}
\usepackage[utf8]{inputenc}
\usepackage{amsmath}
\usepackage{graphicx}
\usepackage{subfig}
\usepackage{listings}
\usepackage{verbatim}

%opening
\title{Trabajo práctico Nº2}
\author{Mariana Vargas Vieyra}

\begin{document}

\maketitle

\begin{abstract}
En este trabajo abordamos dos problemas distintos. En primer lugar estudiamos el efecto de dos 
tratamientos en una muestra de niveles de glucemia. Para ello usamos un modelo mixto generalizado. En segundo lugar, exploramos y modelamos
la producción de leche de un grupo de vacas distribuidas en diferentes tambos. Para este último usamos un modelo no lineal mixto.
\end{abstract}

\section{Problema 1}

Nuestro objetivo para este primer problema es evaluar el efecto de dos programas de alimentación distintos, llámense A y B, en los niveles de glucemia
hallados en medidas repetidas en adultos mayores.

\subsection{Análisis exploratorio}
El comando \verb summary  de \verb R  expone la estructura de los datos y las variables:

\verbatiminput{summary_data.txt}

Añadimos una variable \verb prop  que guarda las proporciones de los conteos por conveniencia. Esto nos permitirá mayor rapidez al graficar y agrega 
información importante a los datos.
En la figura 1 podemos ver el comportamiento de dichas proporciones de las tomas para cada una de las personas, distinguidas por tratamiento. Notar que 
no hay un comportamiento cuadrático marcado con respecto a la variable \verb toma .

\begin{figure}
 \centering
 \includegraphics[width=10cm]{props}
\caption{Gráfico del comportamiento de las proporciones en función de las tomas para cada persona.}
\end{figure}

Un gráfico de mosaico (figura 2) nos muestra que los datos están balanceados.

\begin{figure}[h]
 \centering
 \includegraphics[width=6cm]{mosaic_toma_persona}
\caption{Gráfico de cruzamientos entre persona y toma.}
\end{figure}

\subsection{Modelos}

Hay una decisión de diseño respecto de qué
variables serán tomadas como efecto fijo o aleatorio. Dado que hay sólo dos programas de alimentación siendo evaluados la variable 
\verb Programa.Alimentación  será un efecto fijo. Estamos interesadas en el efecto de estos sobre la población de adultos mayores, por lo tanto es razonable asumir nuestra variable
\verb persona  como un efecto aleatorio cuyas observaciones representan una muestra de una población con distribución normal. Algunas observaciones
al respecto:
\begin{enumerate}
 \item Ignorar el efecto \verb Persona  implicaría una partición del error poco apropiada: parte del error podría ser considerado como aleatorio, los
 niveles de significatividad no serían confiables.
 \item Asumir \verb Persona  como efecto fijo sería erróneo: estaríamos modelando cada una de estas personas y no podríamos extrapolar las conclusiones
 a la población de adultos mayores, que es el objetivo de nuestro estudio.
\end{enumerate}

Ajustamos los siguientes modelos:

\begin{itemize}
 \item El primer modelo evaluado es tal que
 \begin{equation}
  y_{i, j, k} = \beta_{0,i} + \beta_{1,i} \tau_i + \gamma_j + \varepsilon_{i, j, k}
 \end{equation}
en donde $i = A, B$, $j = 1, ..., 10$, y $k = 1, ..., 5$,
$y_{i,j,k}$ es la respuesta de la $j$-ésima persona bajo el $k$-ésimo tratamiento, $\tau_i$ es el efecto del $i$-ésimo tratamiento, que
se asume efecto fijo, $\gamma_j$ es
el efecto aleatorio de la $j$-ésima persona y es tal que $\gamma_j \sim N(0, \sigma_p^2)$, y $\varepsilon_{i,j,k} \sim N(0, \sigma^2)$ el error aleatorio. 

 \item A continuación agregamos \verb toma  como efecto fijo:
 \begin{equation}
  y_{i, j, k} = \beta_{0,i} + \beta_{1,i} \tau_i + \beta_{2, k} t_k + \gamma_j + \varepsilon_{i, j, k}
 \end{equation}
 en donde ahora $t_k$ representa el efecto de la toma $k$-ésima.
\end{itemize}

Para comparar los modelos realizamos un análisis de la varianza con el comando \verb anova , que arroja como resultado que el segundo modelo, el menos
parcimonioso, ajusta mejor los datos. 

\verbatiminput{anova_pr1.txt}

Analizando los resultados del modelo seleccionado observamos que el tratamiento es significativo, así también como las tomas 1 (incluida en el 
intercepto), 4, y 5. La varianza estimada para \verb persona  es $\hat{\sigma_p}^2 = 0.22$.

Recordar que dado que la respuesta consiste en conteos usamos un modelo mixto generalizado con una función link logit, es decir,

\begin{align*}
 X\beta &= \eta(\mu) \\
 &= ln(\frac{\mu}{1 - \mu})
\end{align*}

en donde $\mu = \beta_0 + \beta_1 \tau + \beta_2 T$ para el modelo de la ecuación (2) en forma matricial.
Por lo tanto debemos tomar la inversa de esta función en los resultados arrojados por \verb R  para obtener las estimaciones.

\section{Problema 2}

En esta sección estudiamos la producción de leche de una muestra de vacas distribuidas en varios tambos. La figura 3 nos muestra la tendencia de la cantidad
de litros de leche producidos en función de los días de lactancia para algunas vacas de entre la muestra con la que contamos. Podemos ver que, si bien 
no es muy marcada, hay un comportamiento exponencial negativo en los datos.

\begin{figure}[h]
 \centering
 \includegraphics[width=10cm]{vacas}
\caption{Tendencia de litros de leche producidos en función de los días de lactancia.}
\end{figure}

Un modelo apropiado para evaluar niveles de lactancia es el de Wood, definido de la siguiente manera:

\begin{equation*}
 y = ax^b e^{-cx} + \varepsilon
\end{equation*}

Si ajustamos un modelo de efectos fijos
las estimaciones que arroja \verb R  son $\hat{a} = 19.55$, $\hat{b} = 0.13$, y $\hat{c} = 0.002$. El problema con 
ajustar una curva para todos los sujetos es que las tendencias individuales (que podemos ver en la figura 3) no son debidamente modeladas y se incorporan
al modelo en forma de error aleatorio. El gráfico separado por vaca para residuos vs. predichos de la figura 4 muestra un patrón en su distribución
\footnote{Recordemos que estamos graficando una submuestra de vacas.}. 


\begin{figure}[h]
 \centering
 \includegraphics[width=10cm]{error_vacas_fixed}
\caption{Residuos vs. predichos por vaca.}
\end{figure}

Algo análogo ocurre con los tambos (figura 5).

\begin{figure}[h]
 \centering
 \includegraphics[width=10cm]{error_tambo_fixed}
\caption{Residuos vs. predichos por tambo.}
\end{figure}

Podemos ver en la figura 6 cómo ajusta la curva de este primer modelo a las primeras mil observaciones de nuestros datos.

\begin{figure}[h]
 \centering
 \includegraphics[width=10cm]{fitted_fixed}
\caption{Curva del modelo de efectos fijos.}
\end{figure}

\subsection{Comparación con modelos mixtos}

Para resolver los problemas que acarrea el modelo anterior exploraremos algunas alternativas:

\begin{itemize}
 \item Un modelo que agrupe por tambo, es decir, introduciremos un efecto aleatorio para el tambo. Más específicamente, asumiremos una alteración 
 introducida por el factor \verb TAMBO  en el coeficiente multiplicativo $a$:
 \begin{equation}
  y_{i,j} = (a + \beta_i)x^b e^{-cx} + \varepsilon_{i,j}
 \end{equation}
  en donde $y_{i,j}$ es la cantidad de litros de leche producido por la vaca $j$ en el tambo $i$ en función de los días de lactancia transcurridos, $x$.
  Comparamos este modelo con el de efectos fijos mediante un anova, y concluimos que es necesario modelar el efecto del tambo:
  \verbatiminput{anova_pr2.txt}
 \item Un modelo que agrupe por tambo y a su vez por vaca. Esto es:
 \begin{equation}
  y_{i,j} = (a + \beta_i + \gamma_j)x^b e^{-cx} + \varepsilon{i,j}
 \end{equation}
 Un anova para comparar entre los últimos dos modelos nos permite concluir que este último modelo es el que mejor ajusta los datos. 
 \verbatiminput{anova_pr2_2.txt}
 Notar además que el patrón en los residuos vs. predichos ha mejorado, como se puede ver en la figura 7.
\end{itemize}

\begin{figure}[h]
 \centering
 \includegraphics[width=10cm]{fitted_resid_mod3}
\caption{Residuos vs. predichos para el modelo (4).}
\end{figure}

\subsection{Predicción para 305 días de lactancia}

Comenzamos por ajustar un modelo con un intercepto aleatorio para cada vaca, esto es:
\begin{equation}
  y_{i,j} = (a + \gamma_j)x^b e^{-cx} + \varepsilon_{i,j}
\end{equation}
en donde $\gamma_j$ es el efecto aleatorio de la $j$-ésima vaca. En \verb R , 
\verbatiminput{frand_cow.txt}

Usamos la ecuación 5 para calcular la cantidad de litros de leche esperada para cada día, para cada vaca. Las estimaciones de los parámetros arrojadas
son $a = 14.52$, $b = 0.219$, y $c = 0.003$. El vector $\gamma = (\gamma_1, ... \gamma_v)$, $v = $ número de vacas, puede encontrarse con el comando

\begin{lstlisting}
 re <- random.effects(frand_cow)$a
\end{lstlisting}

Para obtener los totales por vaca agregamos la matriz de litros por día por vaca, \verb predicted , sumando sobre el eje de los días:

\begin{lstlisting}
 accumulated = apply(predicted, 1, sum)
\end{lstlisting}

lo que nos da como resultado la estimación de cantidad de litros total después de 305 días de lactancia para cada vaca.

\subsection{Correlación entre variables observadas y predichas}

Luego de calcular los valores predichos que corresponden a cada uno de los percentiles solicitados, calculamos la correlación entre estos y las variables
observadas. Obtuvimos los siguientes resultados:

\begin{itemize}
 \item \textbf{Percentil 2.5,} correlación de 0.42,
 \item \textbf{Percentil 50,} correlación de 0.73,
 \item \textbf{Percentil 97.5,} correlación de 0.86.
\end{itemize}


\end{document}
