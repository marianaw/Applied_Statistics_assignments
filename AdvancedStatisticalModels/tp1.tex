\documentclass[a4paper,10pt]{article}
\usepackage[utf8]{inputenc}
\usepackage{amsmath}
\usepackage{graphicx}
\usepackage{subfig}
\usepackage{listings}
\usepackage{verbatim}


%opening
\title{TP. 1 MOEA}
\author{Mariana Vargas V. \\ prof. Mónica Balzarini}

\begin{document}

\maketitle

\begin{abstract}
Se nos plantea el problema de decidir qué tipo de plantación, entre varios tipos, es más eficiente en términos de la altura promedio 
alcanzada por árboles de cerezo. En este trabajo desarrollamos el análisis estadístico de los datos y arribamos a una conclusión respecto de qué tratamiento
se adecúa mejor a nuestros objetivos.
\end{abstract}

\section{Descripción de los modelos y salidas (problema 1)}

Hemos analizado cinco modelos que describiremos a continuación. 

\begin{itemize}
 \item \textbf{Primer modelo}. Este modelo puede escribirse como
    \begin{equation}
      y_{i,j,k} = \mu + a_i + t_j + b_k + (at)_{i,j} + (tb)_{j,k} + \epsilon_{i,j,k}
    \end{equation}
  en donde
  \begin{itemize}
   \item $y_{i,j,k}$ es la altura promedio de los árboles en el bloque $k$, bajo el tratamiento $j$, en el año $i$, $k = 1, 2, 3, 4$, $j = 1, ..., 5$, 
   $i = 1, ..., 7$.
   \item $a_i$ es el efecto del $i$-ésimo año y es un efecto fijo,
   \item $t_j$ es el efecto, también fijo, del $j$-ésimo tratamiento,
   \item $b_k$ es el efecto del bloque $k$, también fijo,
   \item $(at)_{i,j}$ es el efecto de la interacción entre el año y el tratamiento, que también es considerado efecto fijo,
   \item $(tb)_{j,k}$ es la interacción entre el tratamiento y el bloque, considerado efecto aleatorio.
   \item $\epsilon_{i,j,k}$ es el residuo.
  \end{itemize}
  
  Recordemos que un efecto aleatorio representa valores tomados aleatoriamente de una distribución de niveles de un factor aleatorio, es decir que 
  en este modelo asumimos que la interacción entre el bloque y el tratamiento son una muestra de una población de este factor con distribución normal
  de media cero y varianzas y covarianzas a estimar. La estructura de varianzas y covarianzas es tal que 
  \[
  G = \\
  \begin{bmatrix}
      \sigma^2_p & \sigma^2_p &  \dots  & \sigma^2_p \\
      \sigma^2_p & \sigma^2_p &  \dots  & \sigma^2_p \\
      \vdots & \vdots &  \ddots & \vdots \\
      \sigma^2_p & \sigma^2_p &  \dots  & \sigma^2_p
  \end{bmatrix}
  \]
  y 
  \begin{equation*}
   R = \sigma^2 I
  \end{equation*}

  en donde $p$ es el factor $bloque * tratamiento$.
  Las salidas sugieren que tanto el tratamiento como el año son significativos, no así la interacción entre ellos. Los bloques tampoco lo son, aunque el 
  efecto aleatorio de interacción entre bloques y tratamiento, si. 
  
 \item \textbf{Segundo modelo.} La ecuación de este modelo es
 \begin{equation}
      y_{i,j,k} = \mu + a_i + t_j + b_k + (at)_{i,j} + \epsilon_{i,j,k}
    \end{equation}
 La principal diferencia con el anterior es que esta vez consideramos los datos como datos longitudinales: con el comando
 \verb repeated  le decimos a SAS que el nivel $tratamiento * bloque$ (que llamaremos $p$ por conveniencia) se repite a través del factor $anio$. Es decir,
 las instancias de esta interacción serán nuestros sujetos. 
 Debido a que elegimos una estructura de covarianzas de simetría compuesta el resultado es prácticamente equivalente al anterior 
 (observar que las estimaciones son casi iguales).
 Lo que ha cambiado es cómo hemos introducido las varianzas. En este caso la matriz $R$ será
  \[
  R = \\
  \begin{bmatrix}
      \sigma^2 + \sigma^2_p & \sigma^2_p &  \dots  & \sigma^2_p \\
      \sigma^2_p & \sigma^2 + \sigma^2_p &  \dots  & \sigma^2_p \\
      \vdots & \vdots &  \ddots & \vdots \\
      \sigma^2_p & \sigma^2_p & \dots  & \sigma^2 + \sigma^2_p
  \end{bmatrix}
  \]
  y $G = 0$. Esto significa que la covarianza de una observación de $p$ en años distintos es aproximadamente $\sigma^2_p$, que dos observariones distintas
  tienen correlación 0, y que la varianza de las observaciones es constante.
  
  \item \textbf{Tercer modelo.} Este modelo tiene la misma ecuación que el anterior, aunque se propone otra estructura de covarianzas: 
  la autorregresiva de orden 1. Esto quiere decir que la matriz
  de varianzas y covarianzas será
  \[
  R = \\
  \begin{bmatrix}
      \sigma^2 & \sigma^2 \rho &  \dots  & \sigma^2 \rho^{n-1} \\
      \sigma^2 \rho & \sigma^2 &  \dots  & \sigma^2 \rho^{n-2} \\
      \vdots & \vdots &  \ddots & \vdots \\
      \sigma^2 \rho^{n-1} & \sigma^2 \rho^{n - 2} &  \dots  & \sigma^2
  \end{bmatrix}
  \]
  de modo que se estimarán los parámetros $\sigma$ y $\rho$. Los criterios de verosimilitud AIC, AICC, y BIC indican que este modelo es el mejor de los que
  hasta ahora mencionamos. Los $p$-valores indican que el tratamiento, el año, y la interacción entre ellos son 
  significativos.
  
  \item \textbf{Cuarto modelo.} Seleccionamos la estrucutra de covarianzas autorregresiva de orden 1, y nos concentramos en ajustar los parámetros del 
  modelo de medias. En este caso tratamos al año como una variable continua e incluimos un término cuadrático, por lo tanto estimamos también coeficientes
  de regresión. La ecuación es:
  \begin{equation*}
   y_{i,j,k} = \mu + c_1 a + t_j + b_k + c_2 a^2 + c_2 (a^2 t_j) + c_1 (at_j) + (tb)_{j,k} + \epsilon_{j,k}
  \end{equation*}
  En la ecuación, $a$ es la covariable año y $c_1$, $c_2$ son coeficientes de regresión a estimar.
  Observar que en este modelo estamos incluyendo el intercepto, lo que significa que estamos ajustando una estructura de medias que tiene la forma de una
  media general más los desvíos. El año, el año al cuadrado, el tratamiento, y la interacción entre estos resultaron significativos. Esto nos dice que 
  los datos tienen un comportamiento cuadrático con respecto al año.
  Además observamos que los criterios de verosimilitud indican que este modelo ajusta mejor que el anterior.
  
  \item \textbf{Quinto modelo.} En este modelo no incluimos intercepto, es decir, modelamos las medias directamente. 
  \begin{equation*}
   y_{j,k} = c_1 a + t_j + b_k + c_2 a^2 + c_2 (a^2 t_j) + c_1 (at_j) + (tb)_{j,k} + \epsilon_{j,k}
  \end{equation*}
  Cada componente representa la media de esa sub-muestra.
  El resultado es un modelo equivalente al anterior, con las mismas estimaciones, con la diferencia de que no estamos modelando desvíos.
\end{itemize}



\section{Modelo elegido (problema 2)}

Lo que observamos es una interacción significativa entre el año y el tratamiento, lo que implica que las alturas de los árboles se comportan de distinta
manera a través de los años para distintos tratamientos. Si hiciécemos un gráfico de promedio de altura vs. años veríamos que las curvas no sería 
paralelas, sino que más bien tenderían a cruzarse en algún momento. Además notamos que la interacción entre el tratamiento y término cuadrático del año es
significativa con coeficientes negativos, lo que sugiere que hay una tendencia cuadrática negativa en la evolución de las alturas con respecto al paso de
los años.
Los modelos que mejor ajustan son el cuarto y el quinto, que asumen una estructura de covarianzas autorregresiva de orden uno y el factor $anio$ como una variable 
continua para la que se ajustan tendencias polinomiales. Lo que asumimos es que las varianzas de los residuos, $\sigma^2$, son constantes, y que la covarianza
de los residuos de dos observaciones separadas por $t$ años es de $\sigma^2 \rho^t$.
Ambos modelos son equivalentes. Podríamos seleccionar el cuarto para entender los factores como desvíos de una media general.

\section{Resultados estadísticos (problemas 3 y 4)}

Podemos concluir que el crecimiento de los árboles está relacionado con el tiempo que transcurre (en años), que, al ser una variable continua, nos provee
de una tasa de crecimiento anual que en este caso es del $68\%$ aproximadamente. 
Los coeficientes de interacción entre el año y el tratamiento, y el año al cuadrado y el tratamiento imponen una ``corrección'' sobre esta tasa anual 
general. En la mayoría de los casos los coeficientes son negativos, excepto en tres, dos de los cuales tienen coeficiente cero. Esto nos dice que el 
crecimiento de los árboles se va desacelerando a medida que pasan los años.

Para decidir qué tratamiento fue mejor al cabo de siete años debemos estudiar las medias de las alturas del último año, en particular si la diferencia
entre ellas resulta significativa.
Las medias del séptimo año son:

\begin{table}[ht]
\centering
\begin{tabular}{rlr}
  \hline
 & Tratamiento & Altura Promedio \\ 
  \hline
  1 & 1 & 3.40 \\ 
  2 & 2 & 3.38 \\ 
  3 & 3 & 3.38 \\ 
  4 & 4 & 3.52 \\ 
  5 & 5 & 3.90 \\ 
   \hline
\end{tabular}
\end{table}

Usamos el test LSD de Fisher para estudiar la diferencia de medias. Ajustamos un modelo lineal sobre los datos del año 7, realizamos un anova y 
corremos el comando \verb LSD.test  del paquete \verb agricolae .
El resultado es:

\begin{table}[ht]
\centering
\begin{tabular}{rlrl}
  \hline
 & trt & means & M \\ 
  \hline
  1 & 5 & 3.90 & a \\ 
  2 & 4 & 3.52 & b \\ 
  3 & 1 & 3.40 & b \\ 
  4 & 3 & 3.38 & b \\ 
  5 & 2 & 3.38 & b \\ 
   \hline
\end{tabular}
\end{table}

Notar que la media del tratamiento 5 es significativamente diferente a las demás, por lo que seleccionamos la plantación de cerezos con alisos en baja
densidad.

\section{Correlación espacial (problemas 5 y 6)}

Ajustamos en R un modelo de correlación espacial usando el comando \verb gls  de la librería \verb nlme . Usamos una estructura gaussiana y una 
exponencial. Al comparar los modelos con el comando \verb anova  obtuvimos:
    
\begin{table}[ht]
\centering
\begin{tabular}{rlrl}
  \hline
  Model & AIC & BIC & logLik \\ 
  \hline
  sin correlación & 2403.314 &  2514.337 &  -1181.657 \\ 
  correlación gaussiana & 2360.601 &  2471.624 &  -1160.300 \\ 
  correlación exponencial & 2334.517 &  2445.541 &  -1147.259 \\ 
   \hline
\end{tabular}
\end{table}

Concluimos que el modelo con correlación espacial exponencial es el que mejor ajusta los datos. Nos concentramos ahora en el modelo con y sin correlación
espacial para los datos de los cerezos de siete años. Usamos correlación gaussiana y comparamos contra el modelo sin correlación:

\begin{lstlisting}
 l_nocorr = gls(Altura ~ Tratamiento + Bloque,
               data = df_Anio7,
               method = "ML",
               na.action = na.omit)

 l_corr = update(l_nocorr, 
		 correlation = corGaus(form = ~Fila + Columna,
				       nugget = FALSE))
 
 anova(l_nocorr, l_corr)
\end{lstlisting}

El resultado de esta comparación arroja:

\verbatiminput{output.txt}

indicando que el modelo con correlación espacial gaussiana ajusta mejor (notar que el test de cociente de verosimilitud es significativo). 
Para verificar qué plantación elegiríamos usamos el test LSD de Fisher nuevamente
a los fines de estudiar las medias:

\begin{lstlisting}
anova(l_corr)  # Esto nos dice que los gl de los residuos es 264.
mean_sq_errors_7 = mean(l_corr$residuals^2); mean_sq_errors_7
lsd_corr_esp_7 = LSD.test(df_Anio7$Altura,
                          df_Anio7$Tratamiento,
                          DFerror = 264,
                          MSerror = mean_sq_errors_7,
                          console = TRUE)
\end{lstlisting}

El resultado es el mismo que antes:

\verbatiminput{LSD_Fisher_A7.txt}

con lo que concluimos que la plantación de cerezos con alisos en baja densidad es la más adecuada alcanzando un promedio de $3.895$ m.
% \begin{figure}
%  \centering
% \includegraphics[width=7cm]{bubble}
% \caption{Gráfico de burbujas para ver la correlación espacial.}
% \end{figure}

\end{document}
