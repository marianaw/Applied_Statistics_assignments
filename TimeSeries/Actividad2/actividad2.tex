\documentclass[a4paper,10pt]{article}
%\documentclass[a4paper,10pt]{scrartcl}
\usepackage[utf8]{inputenc}
\usepackage{amsmath,amssymb}
\usepackage{verbatim}
\usepackage{listings}
\usepackage{mathrsfs}
\usepackage{subcaption}
\usepackage{enumitem}
\usepackage[pdftex]{graphicx}  


\title{Series de Tiempo. Actividad nº2}
\author{Mariana Vargas V.}
\date{\today}

\pdfinfo{%
  /Title    ()
  /Author   ()
  /Creator  ()
  /Producer ()
  /Subject  ()
  /Keywords ()
}

\begin{document}
\maketitle

\section{Primer problema}

El proceso AR(1) está dado por la siguiente ecuación

\begin{equation}
 X_t = 2 + 0.9 X_{t-1} + \varepsilon_t
\end{equation}

Es decir que $C = 2$ y $\phi = 0.9$. 

\begin{enumerate}[label=(\alph*)]
 \item  Si el proceso es estacionario tenemos
 
 \begin{equation*}
  \mu = C + \phi \mu
 \end{equation*}

y por lo tanto

\begin{align*}
 \mu &= \frac{C}{1 - \phi}\\
 &= \frac{2}{1 - 0.9}\\
 &= 20
\end{align*}

La función de varianza será

\begin{align*}
 Var(X_t) &= Var(2 + 0.9X_{t-1} + \varepsilon_t)\\
 &= 0.9^2 Var(X_{t-1}) + 4
\end{align*}

en donde

\begin{equation*}
 Var(X_{t-1}) = \frac{4^2}{1 - 0.9^2}
\end{equation*}

Para calcular la autocovarianza multiplicamos ambos miembros de la ecuación (1) por $X_{t+k}$ y tomamos esperanza:

\begin{align*}
 E(X_t X_{t+k}) &= E(C X_{t+k} + \phi X_t X_{t+k} + \varepsilon X_{t+k}) \\
 \gamma_k &= C E(X_{t+k}) + \phi E(X_t X_{t+k}) + E(\varepsilon) E(X_{t+k}) \text{ , (notar que $\varepsilon$ y $X_{t+k}$ son independientes)} \\
 &= C \mu + \phi \gamma_{k-1} \\
 &= \frac{C^2}{1 - \phi} + \phi^k \gamma_0
\end{align*}


\item En cuanto al proceso en sí, el valor de $\phi = 0.9 < |1|$ indica que es estacionario.

\item La autocorrelación simple está dada por

\begin{align*}
 \rho_k &= \frac{\gamma_k}{\gamma_0} \\
 &= \frac{\frac{C^2}{1 - \phi} + \phi^k \gamma_0}{\gamma_0} \\
 &= \frac{C^2}{(1 - \phi)\gamma_0} + \phi^k
\end{align*}

La autocorrelación parcial $\pi_k$ será nula para todo $k > 1$ dado que el proceso es un AR(1). Por otro lado, sabemos que $\pi_1 = \rho_1$, por lo tanto,

\[ \pi_k = \begin{cases} 
      \frac{C^2}{(1 - \phi)\gamma_0} + \phi^k & k = 1 \\
      0 & k > 1 
   \end{cases}
\]

\end{enumerate}



\section{Segundo problema}

El modelo MA(1) con $\theta = 0.5$ tiene la forma

\begin{equation*}
 X_t = A_t + 0.5 A_{t-1}
\end{equation*}

La autocorrelación teórica de orden 1 está dada por

\begin{align}
 \rho_1(X_t) &= \frac{\theta}{1 + \theta^2}\\
 &= \frac{0.5}{1.5} \\
 &= 0.4
\end{align}

Para muestras grandes el estimador de la autocorrelación, $\hat{\rho_1}$ sigue una distribución normal con media $\rho_1$ y varianza 
$C_{1,1} = (1 - 3\rho^2_1 + 4\rho^4_1)/n$, en donde asumimos $n=1000$. Es decir,

\begin{equation}
 \hat{\rho_1} \sim \mathscr{N}(0.4, 0.0006224)
\end{equation}

Para verificar esto simulamos diez mil procesos MA(1) en R con los parámetros arriba detallados y calculamos las autocorrelaciones de orden 1. Esto nos da como
resultado un arreglo de tamaño 10000 (llamado \verb corrs  en nuestro código) que tiene guardadas cada una de los $\hat{\rho_1}$ de los procesos simulados. 
Calculamos entonces su media y varianza:

\begin{lstlisting}
> mean(corrs)
[1] 0.3983755
> var(corrs)
[1] 0.0006095137
\end{lstlisting}

Observar que coinciden aproximadamente con las media y varianza de la distribución normal descripta en la ecuación (5).

\section{Tercer Problema}

Analizamos a continuación tres series de tiempo dadas. En la figura 1 podemos ver el gráfico de la primera.

\begin{figure}[htp]
 \centering
 \includegraphics[width=.6\textwidth]{ts1}
 \caption{Primera serie de tiempo.}
\end{figure}

En los gráficos de la figura 2 podemos ver cómo tanto la autocorrelación como la autocorrelación parcial están dentro del umbral según el cual podemos decidir si 
una serie de tiempo es ruido blanco, en nuestro caso $2/\sqrt{60} \simeq 0.25$. Luego la serie \verb ts1  fue generada por un proceso de ruido blanco.

\begin{figure}[htp]
\centering
\begin{subfigure}{0.7\textwidth}
\centering
    \includegraphics[width=0.5\textwidth]{ts1_acf}\hfill
    \caption{Autocorrelación.}
    \label{fig:inclu}
\end{subfigure}%
\begin{subfigure}{0.7\textwidth}
\centering
    \includegraphics[width=0.5\textwidth]{ts1_pacf}
    \caption{Autocorrelación parcial.}
    \label{fig:deform}
\end{subfigure}
\caption{Autocorrelaciones de la primera serie.}
\label{fig:manmade}
\end{figure}


La siguiente serie, \verb ts2  , luce como en la figura 3.

\begin{figure}[htp]
 \centering
 \includegraphics[width=.6\textwidth]{ts2}
 \caption{Segunda serie de tiempo.}
\end{figure}

Las autocorrelación decae mientras que la correlación parcial se anula a partir del 2. Esto implica que el proceso fue generado por un modelo AR(1).

\begin{figure}[htp]
\centering
\begin{subfigure}{0.7\textwidth}
\centering
    \includegraphics[width=0.5\textwidth]{ts2_acf}\hfill
    \caption{Autocorrelación.}
    \label{fig:inclu}
\end{subfigure}%
\begin{subfigure}{0.7\textwidth}
\centering
    \includegraphics[width=0.5\textwidth]{ts2_pacf}
    \caption{Autocorrelación parcial.}
    \label{fig:deform}
\end{subfigure}
\caption{Autocorrelaciones de la segunda serie.}
\label{fig:manmade}
\end{figure}

Por último, en la figura 5 podemos ver la forma de la tercera serie propuesta.


\begin{figure}[htp]
 \centering
 \includegraphics[width=.6\textwidth]{ts3}
 \caption{Tercera serie de tiempo.}
\end{figure}

Las autocorrelaciones común y parcial son nulas, sugiriendo que al igual que la primera serie \verb ts3  fue gerada por un modelo de ruido blanco.

\begin{figure}[htp]
\centering
\begin{subfigure}{0.7\textwidth}
\centering
    \includegraphics[width=0.5\textwidth]{ts3_acf}\hfill
    \caption{Autocorrelación.}
    \label{fig:inclu}
\end{subfigure}%
\begin{subfigure}{0.7\textwidth}
\centering
    \includegraphics[width=0.5\textwidth]{ts3_pacf}
    \caption{Autocorrelación parcial.}
    \label{fig:deform}
\end{subfigure}
\caption{Autocorrelaciones de la tercera serie.}
\label{fig:manmade}
\end{figure}


\section{Cuarto Problema}

Observar que la autocorrelación oscila y decae, lo que sugiere que fue un modelo autorregresivo el que generó la serie. En cuanto a la autocorrelación
parcial, podemos
ver que se anula a partir del tercer ``lag'', por lo tanto el proceso es de orden 2.

\end{document}
